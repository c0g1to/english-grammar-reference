\section{Quantifiers}

\subsection{Some, any, no}
\begin{itemize}
    \item Use \underit{some} and \underit{any} to talk about a limited quantity of smth.\\
    Use \underit{not \ldots{} any}, \underit{no} when there is nothing there.
    \item Use \bred{\textit{some/any/no} + plural/uncountable} nouns.
    \item We usually use \underit{some} in:
    \begin{itemize}
        \item \textbf{statements};
        \item questions when we \textbf{expect} the answer \textit{\textbf{yes}},
        especially for offers, requests, and suggestions.
    \end{itemize}
    \item We usually use \underit{any} in:
    \begin{itemize}
        \item \textbf{negative sentences and questions};
        \item statements to mean \textit{it \textbf{doesn't matter} which one}.
    \end{itemize}
    \item We can use \textit{some/any} + \underit{of} + \textit{the-}noun/pronoun.
    \item We can use \underit{some} and \underit{any} \vio{without} a following noun
    when \textbf{it is clear} what they are referring to.
\end{itemize}

\subsection{Smth, everywhere, nobody, anyone}
\begin{itemize}
    \item Use:
    \begin{itemize}
        \item pronouns \underit{-thing} to talk about a thing or an idea;
        \item pronouns \underit{-body} or \underit{-one} to talk about person;
        \item[\daash] \underit{-one} is formal;
        \item adverbs \underit{-where} to talk about a place.
    \end{itemize}
    \item Use:
    \begin{itemize}
        \item \underit{some-} and \underit{every-} in \textbf{statements};
        \item \underit{any-} in \textbf{negatives and questions};
        \item \underit{any-} in statements to mean all when it \textbf{doesn't matter who}, what or where;
        \item \underit{no-} in \textbf{statements and questions}.
    \end{itemize}
    \item[\ast] We don't have two negative words in one sentence.
    \item[\ast] \textit{every-} + singular verb.
\end{itemize}

\subsection{Both, either, neither}
\begin{itemize}
    \item \underit{Both} = A and B. \underit{Neither} = not A and not B. \underit{Either} = A or B\@.
    \item Use \underit{both} + nouns/pronouns, but pronouns + \underit{both}.\\
    Use \underit{either/neither} + singular nouns.
    \item Use \underit{both of} + plural personal pronouns.\\
    Use \underit{either/neither of} + plural nouns and pronouns.
    \item We can use \underit{both \ldots{} and}, \underit{either \ldots{} or}, \underit{neither \ldots{} nor}
    to \textbf{join} nouns, other kinds of words, phrases and even sentences.
    \item We can use \textit{both, either, neither} on their \vio{own} \textbf{as pronouns}.
\end{itemize}

\subsection{All, each, every, none}
\begin{itemize}
    \item[\doot] Use \textit{all, each, every, none} to describe everything in a set.
    \item[\doot] Use \underit{each/every} + singular noun.
    \item[\doot] Use \underit{every} to talk about \textbf{all of a big set}.\\
    Use \underit{each} to mean people or \textbf{things separately}.
    \item[\doot] We can use \textit{All/each/none} + \underit{of} + \textit{the-}noun/pronoun.
\end{itemize}

\subsection{Much, many, little, few}
\begin{itemize}
    \item Use \bred{\textit{much / (a) little / a bit of} + uncountable} nouns;\\
    \bred{\textit{many / (a) few / several / a couple of} + plural} nouns;\\
    \bred{\textit{a lot of / lots of} + both} nouns.
    \item We usually use \underit{much}, \underit{many}, \underit{a lot of} or \underit{lots of}
    in \textbf{negatives and questions}.
    \item \underit{A lot of} or \underit{lots of} are more common in \textbf{informal statements}.
    \item We sometimes use \underit{many} in \textbf{formal statements}.
    \item You can \vio{leave out} the noun after \textit{much, many, a little, a few, a lot of} and \textit{lots of}.
    \item[\doot] We can use \textit{many/much/(a)little/(a)few/most/several} + \underit{of} + \textit{the-}noun/pronoun.
    \item[\doot] \underit{Few} and \underit{little} mean \textbf{not enough}.\\
    \underit{A few} and \underit{a little} mean \textbf{not a lot of, but enough}.
    \item[\doot] Use \underit{plenty of} + uncountable/plural nouns to mean \textbf{enough} or \textbf{more than enough}.
    \item[\doot] \underit{Most} (without \textit{the} and \textit{of}) can mean the \textbf{majority} of.
    \item[\doot] We can use \textit{(too) much / (too) many / (a) little / (a) few
    / several / enough / a lot of / lots of / plenty of / a couple of / a bit of} \textbf{as an adverb}.
    \item[\aast] \textit{A bit} is more informal.
    \item[\aast] \textit{A couple of} means two or three.
\end{itemize}

\subsection{Too and enough}
\begin{itemize}
    \item Use \underit{too} + adjective/adverb to mean \textbf{more than} is reasonable, possible, necessary.
    \item Use adjective/adverb + \underit{enough}.
    \item Use \textit{too much / too many / enough} + noun.
    \item We can \vio{omit} the noun after \textit{enough / too much / too many}.
\end{itemize}