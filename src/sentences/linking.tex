\section{Linking words and sentences}

\subsection{Linking words: and, but, or, so, because}
\begin{itemize}
    \item In a long list, separate the items with commas, but remember to put \textit{and} before the last item.
    \item Use \textit{and} to \textbf{add} information.
    \item Use \textit{but} to \textbf{contrast} information.
    \item Use \textit{or} to show \textbf{alternatives}.
    \item Use \textit{so} to show the \textbf{result} of smth.
    \item Use \textit{because} to show the \textbf{reason} for smth.\\
    If you put \textit{because} \vio{in the beginning}, you need to use a comma.
\end{itemize}

\subsection{Time and sequence adverbs: first, then, afterwards}
\begin{itemize}
    \item Use \textit{first, next, then, afterwards, finally} to describe the \textbf{order} of events.
    \item We can use ordinal numbers to \textbf{describe each stage} of process and \textit{finally} for the last part.
    We usually use commas after these words.
    \item[\ast] We don't usually use \textit{after} as an adverb.
\end{itemize}

\subsection{Word order}
\begin{itemize}
    \item We usually put expressions of time and place and adverbs of manner \vio{in the end}.
    Sometimes we put them \vio{in the beginning}.\\
    If there is more than one of these in the end, the order is usually: \textbf{manner, place, time}.
    \item[\ast] An adverb does not usually come \vio{between} a verb and the object.
\end{itemize}

%TODO linking with TO