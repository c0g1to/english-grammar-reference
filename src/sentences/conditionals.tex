\section{The passive, conditionals}

\subsection{Passive}
\begin{itemize}
    \item Make the passive with \bred{\textit{be} + past participle}.
    \item We usually use the passive when we want to \textbf{focus on the process or result}
    rather than who or what does it.
    \item[\doot] Use \underit{by/with} to mention \textbf{who/what} does it.
    \item[\doot] \underline{Verbs with two objects} have two passive forms:\\
    obj1 + verb in passive + obj2 OR obj2 + verb in passive + obj1.
    \item[\doot] Use verbs which take a \underline{\textit{to-}inf} + \textit{to be} + V3.\\
    Use \underline{adj} + \textit{to be} + V3.\\
    Use verbs which take the \underline{\textit{-ing}} + \textit{being} + V3.
    \item[\doot] Use passive forms of \textit{believe, expect, feel, report, say, think, understand}
    in \underline{reports} where there is some \textbf{uncertainty}.
    \item[\doot] Use \underit{have} + obj + V3 \textbf{to get smbd to do smth}\\
    or that \textbf{smbd has been done smth} (usually bad).\\
    We can use \textit{get} + obj + V3 in informal language.
    \item[\doot] \underit{Need} + \textit{to be} + V3 or \textit{need} + \textit{-ing} has a \textbf{passive meaning}.
    \item[\aast] There are a few verbs describing events or actions which often use \textit{get} instead of \textit{be}.
    \item[\aast] We can use some V3 as adjectives.
\end{itemize}

\subsection{Zero and first conditionals}
\begin{itemize}
    \item In \underline{zero} conditional sentences, use \bred{\textit{if} + present simple + present simple}.
    \item The basic pattern for \underline{first} conditional is:
    \bred{\textit{if} + present simple + future simple}.
    \item Use the \underline{zero} conditional to talk about things that are \textbf{generally true}.
    \item Use the \underline{first} conditional to talk about smth that \textbf{possible in the future}.
    \item When \textit{if} comes \vio{in the beginning} (of any type), we need a \underline{comma} \vio{in the middle}.
    \item We can use \underline{other modals} instead of \textit{will}.
    \item We can use \underit{unless} (in 0 and 1 type) to mean \textbf{\textit{if \ldots{} not}}.
    \item[\doot] We can use (in any type) \textit{as / as long as / provided / providing (that)} \underline{instead of \textit{if}}.
    \item[\ast] Both parts of a first conditional talk about the future.
    \item[\aast] We can use one part of (any type) a conditional sentence in a reply.
    \item[\aast] Use \textit{What if \ldots{}?} for suggestion and speculations.
\end{itemize}

\subsection{Second conditional}
\begin{itemize}
    \item In second conditional, use \bred{\textit{if} + past tense + would + inf}.
    \item Use the second conditional for events and situations which are \textbf{unlikely}, imaginary or impossible.
    \item We often use \textit{\underline{If I were you} \ldots{} I would (not) \ldots} for \textbf{advice} and suggestions.
    \item We often use \textit{if} + \underline{\textit{were} instead of \textit{was}}
    \vio{after} the pronouns \textit{I, she, he, it} and singular nouns.
    This is more common in formal language and American English.
    \item[\doot] We can use \textit{imagine / suppose / supposing (that)} \underline{instead of \textit{if}}.
\end{itemize}

\subsection{\hard{Thrid conditional}}
\begin{itemize}
    \item[\doot] Make with \bred{\textit{if} + \textit{had} + V3 + \textit{would} + \textit{have} + V3}.
    \item[\doot] Use for events in the past which \textbf{did not in fact happen}.
    \item[\doot] Use \underit{I wish / If only} + \textit{had} + V3 to talk about \textbf{regrets}.
    \item[\doot] Use \underit{I wish / If only} + the past simple/continuous with situations
    which \textbf{you would like to be true now}.
    \item[\aast] We can shorten both \textit{had} and \textit{would} to \textit{'d}.
    \item[\aast] \textit{If only} expresses a stronger regret.
\end{itemize}

\subsection{\hard{Variations on conditionals}}
\begin{itemize}
    \item[\doot] Different \underline{combinations of tenses} are possible in conditionals.
    \item[\doot] Any \underline{modals} may be in conditionals.
    \item[\doot] We \bred{don't use \textit{if} + \textit{will/would}} in \textbf{conditionals},
    but we can use it in \textbf{requests} with \textit{if}.
    \item[\doot] Use \textit{if} + \textit{should / happen to} to show that smth is unlikely.
    \item[\doot] We can use \textit{if} + present simple + \underline{imperative}.
    \item[\doot] Use \textit{if} + \textit{wasn't/weren't/hadn't} + \textit{for}
    to show that one \textbf{thing changes the situation completely}.
    \item[\doot] We can put \textit{had / should / were}+pronoun/noun instead of \textit{if} \vio{in the beginning}.
    This is more \textbf{formal}.
    \item[\aast] We can use \textit{then} in the main part for emphasis.
\end{itemize}

