\section{Adjectives and adverbs}

\subsection{Adjectives}
\begin{itemize}
    \item Use \bred{adjective + noun}.
    \item Use \bred{\textit{be, feel, look, seem, smell, sound, taste} + adjective}.
    \item[\doot] There are a few adjectives which we only use \vio{after} the verb:
    \textit{afraid, alive, alone, asleep, awake, glad, ill, well}.
    \item A number of adjectives \underline{end in} \textit{-y}; \textit{-ful} means \textbf{full of},
    \textit{-less} means \textbf{without}.
    \item[\doot] We can use singular \underline{nouns} as adjectives.
    \item[\doot] We can use some adj + \textit{to} + \underline{inf}.
\end{itemize}

\subsection{-ed and -ing adjectives}
\begin{itemize}
    \item We can use \textit{-ing, -ed} or V3 form of the verb as adjectives.
    \item Adjectives which can end in \underline{either} \textit{-ing} or \textit{-ed}:
    \begin{itemize}
        \item \underit{-ed} endings describe how \textbf{we feel}.
        \item \underit{-ing} endings describe what or who \textbf{causes the feeling}.
    \end{itemize}
\end{itemize}

\subsection{Gradable adjectives}
\begin{itemize}
    \item Use \underit{very / quite} + gradable adjective.
    \item We can use \underit{a bit} / \underit{a little} + gradable/comparative adjective or on it's \vio{own}.
    \item[\doot] Don't use these words \vio{in front of}
    \textit{married, delicious, dead, enormous, exhausted, impossible, perfect},%todo add remaining or remove
    which mean smth \textbf{absolute} or extreme.
    \\We can, however, use words \vio{like} \textit{absolutely, completely, totally, extremely}.
    \item[\doot] \underit{quite} + gradable adj = \textbf{fairly};\\
    \underit{quite} + upgradable adj = \textbf{completely}.
    \item[\aast] We use \textit{quite} + \textit{a} + adj + noun.
\end{itemize}

\subsection{Order of adjectives}
\begin{itemize}
    \item The usual order of adjectives is: \bred{opinion, size, quality, age, shape, colour, origin, material, purpose}.
    \item We often use \underit{hyphens} in an adjective phrase \vio{with} numbers.
\end{itemize}

\subsection{Comparatives}
\begin{itemize}
    \item Use comparative adjective + \textit{than} to \textbf{compare} things.
    \item Don't use \textit{very} \vio{before} comparative adjective.
    Use \textit{much, far, a lot, a little, a bit}.\\
    \textit{A lot} and \textit{a bit} is more informal.
\end{itemize}

\subsection{Superlatives}
\begin{itemize}
    \item Use \bred{\textit{the} + superlatives}.
    \item We can use superlatives \vio{without} a noun.
\end{itemize}

\subsection{Adverbs of manner}
\begin{itemize}
    \item We usually form adverbs of manner by \bred{adding \textit{-ly}} to the adjective.
    \item To make the adverb from some adjectives end in \textit{-ly}, we say \bred{\textit{in a \ldots{} way}}.
    \item Some adverbs are the same as the adjectives.
    \item Adverbs of manner often come \vio{in the end}.
\end{itemize}

\subsection{Comparative and superlative adjectives and adverbs}
\begin{itemize}
    \item Use \bred{\textit{more}/\textit{less} + adjectives/adverbs} to make the comparative form and\\
    \bred{\textit{the} + \textit{most}/\textit{least}  + adjectives/adverbs} to make the superlative form.
    \item Many short adjectives and adverbs have comparative forms with \bred{\textit{-er}}
    and superlative forms with \bred{\textit{-est}}.
    \item Some common adjectives and adverbs have irregular comparative and superlative forms.
    \item Use \textit{as} + adjective/adverb + \textit{as} to say that things are \textbf{equal}.
    \item[\ast] We often use \textit{not as \ldots{} (as)} instead of \textbf{\textit{less}} in informal language.
\end{itemize}

\subsection{Adverbs of frequency}
\begin{itemize}
    \item Adverbs of frequency usually go
    \vio{after} the verb \textit{be}, auxiliaries, modals, \textit{not}; \vio{before} the main verb.
    \item[\ast] \textit{Sometimes} goes \vio{before} the verb \textit{be}, auxiliaries, modals, \textit{not} in negatives.
    \item[\ast] \textit{Usually, normally, often} and \textit{sometimes}
    can also go \vio{in the beginning} or \vio{in the end}.
\end{itemize}