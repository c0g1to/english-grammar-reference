\section{Verbs}

\subsection{Have}
\begin{itemize}
    \item We can use \textit{have} + noun for many \textbf{everyday activities}.\\
    \textit{Have} with activities can be in the continuous form.
\end{itemize}

\subsection{Phrasal verbs}
\begin{itemize}
    \item Phrasal verbs have two words: \bred{verb + adverb}.\\
    Some phrasal verbs have an object and some don't.
    \item \underline{Object} can go \vio{before and after} the adverb particle.\\
    If object is a personal pronoun, it always comes \vio{before} the adverb particle.
    \item[\ast] Prepositional and phrasal verbs are informal and one-word verbs is more formal.
    \item[\aast] There are some nouns which are based on phrasal verbs.
\end{itemize}

\subsection{Gerund; would like}
\begin{itemize}
    \item Some verbs take \underline{another verb} with \textit{-ing} or \textit{to-}inf or any.
    \item Some verbs take both forms with \textbf{difference in meaning}.\\
    \underit{To} forms usually have \textbf{active} meaning, \underit{-ing} are \textbf{passive} or \textbf{past}.
    \item Use \bred{\textit{would like} + \textit{to}-inf}
    for a \textbf{polite invitation} or for \textbf{saying \textit{want}}.
    \item[\doot] Some verbs take an \underline{object and} \textit{to-}inf or just inf.
    \item[\aast] In speaking, \textit{get} + object + \textit{to-}inf = \textit{persuade}.
\end{itemize}

\subsection{Get, make, do}
\begin{itemize}
    \item Use \underline{\textit{get} + object} to mean \textbf{receive} or obtain.
    \item Use \underit{make} to talk about:
    \begin{itemize}
        \item \textbf{producing} smth;
        \item being the \textbf{cause of someone's feeling}.
    \end{itemize}
    \item Use \underit{do} to talk about work and \textbf{activities}.
\end{itemize}

\subsection{\hard{Linking verbs}}
\begin{itemize}
    \item[\doot] All linking verbs \bred{can be followed by an adjective},\\
    but \textit{be, become, feel, look, remain, stay, sound} \bred{can also be followed by nouns}.
    \item[\doot] Some describe things that \textbf{change}: \textit{become, get, go, grow, turn}:
    \begin{itemize}
        \item[\daash] use \underit{turn} and \underit{go} with \textbf{colours};
        \item[\daash] \underit{go} describes \textbf{bad} changes.
        \item[\daash] use \underit{get} or \underit{become} (not \textit{go}) \vio{with} \textit{old, tired, ill};
        \item[\daash] use \underit{get} (not \textit{become}) in \textbf{imperatives} and for \textbf{shorter process};
        \item[\aast] \textit{go} and \textit{get} usually are more informal.
    \end{itemize}
    \item[\doot] Some mean \textbf{staying the same}: \textit{keep, remain, stay}.
    \item[\doot] Some describe \textbf{senses}: \textit{appear, feel, look, seem, smell, sound, taste}.
    \item Use \bred{\textit{What does it look/feel/\ldots{} like?}} or \bred{\textit{What is it like?}}
    to ask \underline{questions} \textbf{about the senses}.
    The \underline{answer} has \bred{linking verb + adjective / \textit{like}+noun}.
    \item[\doot] Prepositional verbs like \bred{\textit{look/feel/\ldots{}} + \textit{like} + noun} mean \textbf{resemble}.
    \item[\aast] A few descriptive verbs, e.g. \textit{lie, fall, sit, stand} can sometimes be linking.
    \item[\aast] Adjectives, beginning with \textit{a-} and \textit{ill, well} have status meaning.
\end{itemize}

\subsection{Verbs with two objects}
\begin{itemize}
    \item Some verbs have two objects: \bred{subject + verb + direct + \textit{to/for} + indirect}.
    \item Use \underit{for} \vio{before} indirect object \vio{with} \textit{build, buy, find, get, leave, make}.
    \item Use \underit{to} \vio{before} indirect object in other cases.
    \item[\doot] There is \underline{no preposition} \vio{with} \textit{allow, charge, cost, fine, wish}.
    \item We can also put: \bred{subject + verb + indirect + \textit{(to/for)} + direct}.
\end{itemize}

