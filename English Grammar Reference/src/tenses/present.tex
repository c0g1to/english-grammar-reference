\section{Present}

\subsection{Be}
\begin{itemize}
    \item Use \textit{be}:
    \begin{itemize}
        \item with ages;
        \item with \textit{a/an} + jobs;
        \item to describe the weather;
        \item to talk about \textbf{time and place}.
    \end{itemize}
\end{itemize}

\subsection{Present simple}
\begin{itemize}
    \item Use \bred{\textit{I/you/we/they} + inf}
    or \bred{\textit{he/she/it} + \textit{-s}}.
    \item Use the present simple:
    \begin{itemize}
        \item for things that are always or \textbf{usually true};
        \item for regular or \textbf{repeated} events and habits;
        \item for \textbf{states}, \vio{with} verbs such as
        \textit{believe, feel, hate, know, like, look, love, mean, prefer, promise, sound, think, understand, want};
        \item we often use it \vio{with} adverbs of frequency \textit{always, never, sometimes, usually}\ldots
        \item[\daash] to give \textbf{instructions} or directions;
        \item[\daash] to tell a story, or to describe \textbf{series of events};
        \item[\daash] for a \textbf{review}.
    \end{itemize}
    \item There is no \textit{-s} on the main verb \vio{after} \textit{does} or \textit{doesn't}.
\end{itemize}

\subsection{Present continuous}
\begin{itemize}
    \item Make statement with \bred{\textit{be} + \textit{-ing}}.
    \item Use the present continuous:
    \begin{itemize}
        \item for activity \textbf{in progress now} or around now;
        \item for \textbf{unfinished temporary} actions;
        \item[\daash] \vio{with} \textit{always, constantly, forever} to \textbf{criticise};
        \item[\daash] for situations which are \textbf{gradually changing};
        \item[\daash] to describe the \textbf{background} to a story;
        \item[\daash] to \textbf{emphasize the process} \vio{using} state verbs such as
        \textit{appear, expect, feel, have, imagine, look, think, see, smell, taste, weigh};
        \item[\aast] in informal letters and emails.
    \end{itemize}
\end{itemize}

\subsection{Imperatives}
\begin{itemize}
    \item Use the \bred{inf} for imperatives.
    \item \textit{Always} and \textit{never} can be used \vio{at the beginning of} the imperative phrase.
    \item There is usually no subject.
    \item Imperatives sound not very polite.
\end{itemize}

\subsection{Present perfect}
\begin{itemize}
    \item Make using the verb \bred{\textit{have} + V3}.
    \item Use the present perfect:
    \begin{itemize}
        \item to talk about a \textbf{recent event}. Use \textit{just} to emphasize it;
        \item for past event which the speaker feels is \textbf{connected with the present};
        \item when it is \textbf{not important when} the past event took a place;
        \item for \textbf{states} which \textbf{started in the past and continues now}.
    \end{itemize}
    \item We can use \underit{already} and \underit{yet}. They mean 'before now'.\\
    We use \underit{yet} in question and negatives. \underit{Yet} comes \vio{in the end}.\\
    \underit{Already} comes \vio{after} \textit{has/have}.
    \item We often use \underit{ever}, \underit{never} and \underit{before}.\\
    \underit{Never} and \underit{ever} come \vio{after} \textit{has/have}.\\
    \underit{Before} comes \vio{in the end}.
    \item Use \underit{since} to emphasize when a situation \textbf{began}.
    \item Use \underit{for} to emphasize \textbf{how long} a situation has been going.
    \item \underit{Still} stresses that the situation is \textbf{continuing now}.
    \item[\ast] We can use \textit{always}.
\end{itemize}

\subsection{\hard{Present perfrect continuous}}
\begin{itemize}
    \item[\doot] Make using \bred{\textit{have} + \textit{been} + \textit{-ing}}.
    \item[\doot] We use the present perfect continuous for:
    \begin{itemize}
        \item[\daash] \textbf{activity} that is still \textbf{going on} or has only \textbf{just ended}\\
        (usually for shorter temporary);
        \item[\daash] \textbf{global changes}.
    \end{itemize}
    \item[\aast] Often there is no difference between the perfect and the perfect continuous.
\end{itemize}