\section{Reported speech, relative clauses}

\subsection{Reported speech}
\begin{itemize}
    \item If the verb of saying or thinking is in the present,
    there is \bred{no change} of tense for the words reported.
    \item When the verb of saying or thinking is in the past,
    the \bred{verb} in the reported speech usually \bred{moves into the past}.
    \item The verb in the reported speech \bred{does not need to change}
    if the information is still true or \textbf{relevant} now.
    \item Pronouns and time and place expressions \bred{may change} in reported speech.
    \item \textit{That} often \vio{links} the verb of saying or thinking to the reported speech.
    \textit{That} can be \vio{left out} in informal language.
    \item[\ast] \textit{Used to} and \textit{would} do not change in reported speech.
\end{itemize}

\subsection{Say, tell, speak, talk}
\begin{itemize}
    \item Use \underit{say}:
    \begin{itemize}
        \item when it is not necessary to \textbf{specify who} is being spoken to;
        \item to introduce \textbf{direct speech}.
    \end{itemize}
    \item Use \underit{tell}:
    \begin{itemize}
              \item + noun to give \textbf{information};
              \item + object + \textit{to} + inf to report instructions or \textbf{commands}.
    \end{itemize}
    \item There are a number of expressions using \underit{tell} + noun and some with \underit{say}.
    \item Use \underit{speak} to talk about the \textbf{ability} to speak.
    \item Use \underit{talk} to mean \textbf{have a conversation}.
\end{itemize}

\subsection{Defining relative clauses}
\begin{itemize}
    \item Use \textit{who} to refer to \textbf{person}.\\
    Use \textit{which} to refer to a \textbf{thing, an animal or an idea}.\\
    Use \textit{that} or \textit{who} or \textit{which} in informal English.
    \item We can \vio{leave out} the relative pronoun when it is the object of the relative clause.
\end{itemize}