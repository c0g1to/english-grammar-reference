\section{The passive, conditionals}

\subsection{Passive}
\begin{itemize}
    \item Make the passive with \bred{\textit{be} + past participle}.
    \item We usually use the passive when we want to \textbf{focus on the process or result}
    rather than who or what does it.
    \item[\doot] We can use \underit{by/with} to mention \textbf{who/what} does it.
    \item[\aast] There are a few verbs describing events or actions which often use \textit{get} instead of \textit{be}.
\end{itemize}

\subsection{Zero and first conditionals}
\begin{itemize}
    \item In zero conditional sentences, use \bred{\textit{if} + present tense \ldots{} present tense}.
    \item The basic pattern for first conditional is:
    \bred{\textit{if} + present tense \ldots{} \textit{will} + infinitive without to}.
    \item Use the zero conditional to talk about things that are \textbf{generally true}.
    \item Use the first conditional to talk about smth that we think is \textbf{possible in the future},
    and its result.
    \item Use \textit{might} or \textit{could} in the main part to indicate (any type)
    that smth is possible and \textbf{not certain}.
    \item We can use \textit{unless} to mean \textbf{\textit{if \ldots{} not}}.
    \item When \textit{if} comes \vio{in the beginning} (any type), we need a comma \vio{in the middle}.
    \item[\ast] Both parts of a first conditional talk about the future.
\end{itemize}

\subsection{Second conditional}
\begin{itemize}
    \item In second conditional, use \bred{\textit{if} + past tense \ldots{} would + infinitive without to}.
    \item Use the second conditional for events and situations
    which are \textbf{unlikely}, imaginary or impossible in the present or future.
    \item We often use \textit{If I were you \ldots{} I would (not) \ldots} for \textbf{advice} and suggestions.
    \item We often use \textit{if} + \textit{were} instead of \textit{was}
    \vio{after} the pronouns \textit{I, she, he, it} and singular nouns.
    This is more common in formal language and American English.
\end{itemize}