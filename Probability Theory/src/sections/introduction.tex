\section{Введение}

\subsection{Пространство элементарных событий}
$(\Omega, \frak{A}, P)$ ~--- \bred{вероятностное пространство};\\
$\Omega$ --- \bred{пространство элементарных событий};\\
$\omega$ --- \bred{элементарные события}, соответствуют взаимоисключающие исходы;\\
$\Omega = \{\omega_1, \omega_2, \ldots, \omega_n\}$.

\bred{Случайное событие} --- любое подмножество $\Omega$, если оно конечно или счетно;\\
\textbf{Счетное множество} --- элементы которого можно занумеровать.\\
$\Omega$ --- \textbf{достоверное} событие;\\
$\emptyset$ --- \textbf{невозможное} событие.\\
$A$ и $B$ --- несовместны, если $AB = \emptyset$.

\subsection{Алгебра событий}
$\frak{A}$ - класс подмножеств множества $\Omega$;\\
$\frak{A}$ - \bred{алгебра событий},\\
если $\Omega \in \frak{A},\ \emptyset \in \frak{A},\ AB \in \frak{A},
\ A + B \in \frak{A},\ A \setminus B \in \frak{A}$\\
при любых $A \in \frak{A}$ и $B \in \frak{A}$.\\

Если $\Omega$ --- конечное, тогда $\frak{A}$ --- конечное;\\
Если размерность $\Omega$ --- $N$, тогда число всех подмножеств $2^N$.\\

Алгебра событий $\frak{A}$ --- \bred{$\sigma$-алгебра} (борелевская алгебра), если\\
из $A_n \in \frak{A},\ n = 1,2\ldots$, следует$ \quad \bigcup_{n=1}^{\infty} A_n \in \frak{A} \quad
\text{и} \quad \bigcap_{n=1}^{\infty} A_n \in \frak{A}$.

\subsection{Вероятность}
Числовая функция $P$, определенная на классе событий $\frak{A}$ - \bred{вероятность}, если:
\begin{enumerate}
    \item $\frak{A}$ является алгеброй событий.
    \item $P(A) \geq 0$ для любого $A \in \frak{A}.$
    \item $P(\Omega) = 1.$
    \item Если $A$ и $B$ \bred{несовместны}, то $P(A + B) = P(A) + P(B)$\\
    (\textbf{аксиома конечной аддитивности}).\\
    \textbf{Несовместные события} --- те, которые не могут произойти одновременно.
    \item Для любой последовательности $A_1 \supset A_2 \supset \ldots \supset A_n$ из $\frak{A}$ такой,\\
    что $\bigcap_{n=1}^{\infty} A_n = \emptyset$,\\
    справедливо $\lim_{n \to \infty} P(A_n) = 0$.
\end{enumerate}

Свойства:
\begin{itemize}
    \item $P(\overline{A}) = 1 - P(A)$.
    \item $P(A + B) = P(A) + P(B) - P(AB)$.
    \item Если события $A_1, A_2, \dots, A_n$ --- попарно несовместны
    и $A = \bigcup_{n=1}^{\infty} A_n \in \frak{A}$,\\
    то $P(A) = \sum_{n=1}^\infty P(A_n)$.
    \item Если $A_1 \subset A_2 \subset \ldots \subset A_n \quad \text{и} \quad A=\bigcup_{n=1}^{\infty} A_n$\\
    или $A_1 \supset A_2 \supset \ldots \supset A_n \quad \text{и} \quad A=\bigcap_{n=1}^{\infty} A_n$,\\
    то $P(A) = \lim_{n \to \infty} P(A_n)$.
\end{itemize}