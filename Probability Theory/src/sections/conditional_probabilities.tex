\section{Условные вероятности}

\subsection{Условные вероятности}
\bred{Условная вероятность} $P(B|A)$ события $B$ при условии, что $A$ произошло
\begin{equation*}
    P(B|A)=\frac{|AB|}{|A|}=\frac{P(AB)}{P(A)}
\end{equation*}
Если $A$ и $B$ \bred{независимы}, то
\begin{equation*}
    P(B|A)=P(B) \quad \text{и} \quad P(AB)=P(A) \cdot P(B)
\end{equation*}
\textbf{Независимые события} --- наступление одного, не изменяет вероятность наступления другого.

\subsection{Вероятность произведения событий}
\bred{Теорема умножения}:
\begin{equation*}
    P(A_1 A_2 \ldots A_n) = P(A_1) P(A_2 | A_1) \ldots P(A_n | A_1 \ldots A_{n-1})
\end{equation*}
События $A_1, \ldots, A_n$ \bred{взаимно независимые}, если для всех комбинаций:
\begin{equation*}
    P(A_1 A_2 \ldots A_n) = P(A_1) P(A_2) \ldots P(A_n)
\end{equation*}

\subsection{Формула полной вероятности}
Пусть $A$ --- произвольное событие, $B_1, B_2, \ldots, B_n$ попарно несовместны и
$A \subset B_1 + B_2 + \ldots + B_n$, тогда \bred{формула полной вероятности}:
\begin{equation*}
    P(A) = \sum_{i = 1}^n P(B_i) P(A|B_i)
\end{equation*}
\bred{Формула Байеса}:
\begin{equation*}
    P(B_k|A) = \frac{P(B_k)P(A|B_k)}{\sum_{i = 1}^n P(B_i) P(A|B_i)}
\end{equation*}